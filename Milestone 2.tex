\documentclass{article}
\usepackage{url}
\usepackage{polski}
\usepackage[utf8]{inputenc}
\usepackage{color}

\title{Milestone 2}
\author{Mirosław Kuźniar, Szymon Porzeziński, Bartosz Skrzypczak }
\date{\today}

\begin{document}

\maketitle

\section{Opis w formie słownej}
Zadaniem symulacji jest przeprowadzenie startu oraz lotu rakiety wraz z uwzględnieniem wielkości wpływających. Sama rakieta będzie składać się z komponentów/elementów składowych – takich jak między innymi silnik czy zbiornik paliwa od których parametrów będzie zależał przebieg symulacji. Ponadto uwzględnione zostaną także zewnętrze czynniki, takie jak między innymi wiatr, które także będą miły wpływ na przebieg symulacji. Za zakończenie symulacji uznany będzie moment w którym rakieta osiągnie zadany pułap. 

\section{Analiza czasownikowo – rzeczownikowa}
Potrzeby będzie obiekt – rakieta. Na rzecz którego zostanie przeprowadzona symulacja. Będzie on zawierał \textcolor{blue}{obiekty składowe}, których atrybuty będą miały wpływ na przebieg symulacji. Chcemy, aby \textcolor{yellow}{umożliwiały one wyłuskanie} \textcolor{blue}{ilości paliwa zatankowanego do rakiety a także ciągu silników}. W symulacji będzie istniał także obiekt odpowiadający za symulację czynników zewnętrznych. Szczególnym przypadkiem tego obiektu będzie \textcolor{blue}{obiekt „wiatr”}, którego atrybuty także będą miały wpływ na przebieg symulacji. Chcemy, aby \textcolor{yellow}{umożliwiał on wyłuskanie} \textcolor{blue}{kierunku oraz siły wiatru}. Potrzebny będzie \textcolor{blue}{obiekt}, który \textcolor{yellow}{w oparciu o atrybuty wyżej wspominanych obiektów będzie w stanie wyznaczyć} \textcolor{blue}{tor oraz prędkość lotu} rakiety. Potrzeby będzie także  \textcolor{blue}{obiekt}, \textcolor{yellow}{który sprawdzi czy rakieta osiągnęła} \textcolor{blue}{zadany pułap} i zakończy symulację. 

\section{Diagram przypadków użycia}

\section{Karty CRC}

\begin {center}

\begin{tabular}{|c|c|}
\hline
\multicolumn{2}{|l|}{\textbf{Classname:} Rocket}\\
\hline
\multicolumn{2}{|l|}{\textbf{Superclass:} none}\\
\multicolumn{2}{|l|}{\textbf{Subclass/es:} none}\\
\hline
\parbox[]{5cm}{\vspace{3px}\textbf{Responsibilities:} \\Coordinates interactions between its parts, and stores the current state of the rocket (position, velocity, direction)\vspace{3px}} & \parbox[]{5cm}{\textbf{Collaborations:}\\Engine\\Tank\\Vector}\\
\hline
 \end{tabular}\vspace{.4cm}
 
\begin{tabular}{|c|c|}
\hline
\multicolumn{2}{|l|}{\textbf{Classname:} Engine}\\
\hline
\multicolumn{2}{|l|}{\textbf{Superclass:} IRocketPart}\\
\multicolumn{2}{|l|}{\textbf{Subclass/es:} none}\\
\hline
\parbox[]{5cm}{\vspace{3px}\textbf{Responsibilities:} \\Based on its own thrust values and the amount of remaining fuel, creates thrust force\vspace{3px}} & \parbox[]{5cm}{\textbf{Collaborations:}\\FuelTank\\Vector}\\
\hline
 \end{tabular}\vspace{.4cm}

\begin{tabular}{|c|c|}
\hline
\multicolumn{2}{|l|}{\textbf{Classname:} Tank}\\
\hline
\multicolumn{2}{|l|}{\textbf{Superclass:} IRocketPart}\\
\multicolumn{2}{|l|}{\textbf{Subclass/es:} none}\\
\hline
\parbox[]{5cm}{\vspace{3px}\textbf{Responsibilities:} \\Keeps track of the amount (mass) of the remaining fuel\vspace{3px}} & \parbox[]{5cm}{\textbf{Collaborations:}\\none}\\
\hline
 \end{tabular}\vspace{.4cm}

\begin{tabular}{|c|c|}
\hline
\multicolumn{2}{|l|}{\textbf{Classname:} Wind}\\
\hline
\multicolumn{2}{|l|}{\textbf{Superclass:} IForceField}\\
\multicolumn{2}{|l|}{\textbf{Subclass/es:} none}\\
\hline
\parbox[]{5cm}{\vspace{3px}\textbf{Responsibilities:} \\Creates a force based on wind speed and direction at a given point\vspace{3px}} & \parbox[]{5cm}{\textbf{Collaborations:}\\Vector}\\
\hline
 \end{tabular}\vspace{.4cm}

\begin{tabular}{|c|c|}
\hline
\multicolumn{2}{|l|}{\textbf{Classname:} GravityField}\\
\hline
\multicolumn{2}{|l|}{\textbf{Superclass:} IForceField}\\
\multicolumn{2}{|l|}{\textbf{Subclass/es:} none}\\
\hline
\parbox[]{5cm}{\vspace{3px}\textbf{Responsibilities:} \\Calculates the force of gravity, based on position\vspace{3px}} & \parbox[]{5cm}{\textbf{Collaborations:}\\Vector}\\
\hline
 \end{tabular}\vspace{.4cm}

\begin{tabular}{|c|c|}
\hline
\multicolumn{2}{|l|}{\textbf{Classname:} AirResistanceField}\\
\hline
\multicolumn{2}{|l|}{\textbf{Superclass:} IForceField}\\
\multicolumn{2}{|l|}{\textbf{Subclass/es:} none}\\
\hline
\parbox[]{5cm}{\vspace{3px}\textbf{Responsibilities:} \\Calculates the force of air resistance, based on position and speed\vspace{3px}} & \parbox[]{5cm}{\textbf{Collaborations:}\\Vector}\\
\hline
 \end{tabular}\vspace{.4cm}

\begin{tabular}{|c|c|}
\hline
\multicolumn{2}{|l|}{\textbf{Classname:} IForceField}\\
\hline
\multicolumn{2}{|l|}{\textbf{Superclass:} none}\\
\multicolumn{2}{|l|}{\textbf{Subclass/es:} Wind, GravityField, AirResistanceField}\\
\hline
\parbox[]{5cm}{\vspace{3px}\textbf{Responsibilities:} \\A general interface for interacting with force fields\vspace{3px}} & \parbox[]{5cm}{\textbf{Collaborations:}\\Vector}\\
\hline
 \end{tabular}\vspace{.4cm}

\begin{tabular}{|c|c|}
\hline
\multicolumn{2}{|l|}{\textbf{Classname:} Vector}\\
\hline
\multicolumn{2}{|l|}{\textbf{Superclass:} none}\\
\multicolumn{2}{|l|}{\textbf{Subclass/es:} none}\\
\hline
\parbox[]{5cm}{\vspace{3px}\textbf{Responsibilities:} \\Storage and operations on vectors\vspace{3px}} & \parbox[]{5cm}{\textbf{Collaborations:}\\none}\\
\hline
 \end{tabular}\vspace{.4cm}

\begin{tabular}{|c|c|}
\hline
\multicolumn{2}{|l|}{\textbf{Classname:} IRocketPart}\\
\hline
\multicolumn{2}{|l|}{\textbf{Superclass:} none}\\
\multicolumn{2}{|l|}{\textbf{Subclass/es:} Engine, Tank}\\
\hline
\parbox[]{5cm}{\vspace{3px}\textbf{Responsibilities:} \\Can have mass and may be a source of a force\vspace{3px}} & \parbox[]{5cm}{\textbf{Collaborations:}\\Vector}\\
\hline
 \end{tabular}\vspace{.4cm}

\begin{tabular}{|c|c|}
\hline
\multicolumn{2}{|l|}{\textbf{Classname:} World}\\
\hline
\multicolumn{2}{|l|}{\textbf{Superclass:} none}\\
\multicolumn{2}{|l|}{\textbf{Subclass/es:} none}\\
\hline
\parbox[]{5cm}{\vspace{3px}\textbf{Responsibilities:} \\Controls the simulation, determines when to start and stop it, and shows the results\vspace{3px}} & \parbox[]{5cm}{\textbf{Collaborations:}\\Rocket\\IForceField}\\
\hline
 \end{tabular}\vspace{.4cm}

\end{center}

\section{Diagramy klas}
\end{document}
